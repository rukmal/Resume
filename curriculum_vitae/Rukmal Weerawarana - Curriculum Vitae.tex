\documentclass[10pt]{article}

% Package Imports
%----------------

\usepackage{enumitem}  % Advanced list settings
\usepackage{fancyhdr}  % Fancy headers and footers
\usepackage[T1]{fontenc}  % Expanded font encoding
\usepackage{geometry}  % Page geometry (margins, etc.)
\usepackage{hyperref}  % Hyperlinks
\usepackage[none]{hyphenat}  % Hyphenation settings
\usepackage{multirow}  % Multirow table
\usepackage[defaultsans]{opensans}  % Opensans font
\usepackage[document]{ragged2e}  % Text alignment
\usepackage{tabularx}  % Advanced tables
\usepackage{titlesec}  % Title formatting
\usepackage[svgnames]{xcolor}  % Colors

% LaTeX Configuration
%---------------------

% Page margins
\geometry{top=.5in,
          bottom=.5in,
          left=.5in,
          right=.5in}

% Document font size
\newcommand{\cvfontsize}{10}

% Document font stuff
\renewcommand{\familydefault}{\sfdefault}
\renewcommand{\normalsize}{\fontsize{\cvfontsize}{\baselineskip}\selectfont}

% Highlight color
\newcommand{\highlightcolor}{RoyalBlue}

% Reformat section
\titleformat{\section}[block]
{\color{\highlightcolor} \Large \bf \scshape }
{}{0em}{}

% No indentation
\setlength{\parindent}{0pt}

% Table width
\newcommand{\tabularxwidth}{\textwidth}

% Link formatting
\hypersetup{
    colorlinks=true,
    linkcolor=blue,
    urlcolor=blue
}


% Header and Footer Configuration
%--------------------------------

% Changing page style to fancy
\pagestyle{fancy}

% Applying default headers and footers
\fancyhf{}

% Renewing commands for the header and footer lines
\renewcommand{\headrulewidth}{0pt}
\renewcommand{\footrulewidth}{0pt}

% Changing offset from the margins
\setlength{\footskip}{0pt}

% Setting footer to be "Page #" on bottom right
\rfoot{Page \thepage}

% Setting last page footer to include last updated date, and Precis signature.
\AtEndDocument{\lfoot{Last updated December 15, 2021. Built with Precis (\url{https://precis.rukmal.me}).}}


% Jinja Macros
%--------------

% Get state or country (state takes preference)

% Get location in "City, State/Country" format

% Format date (to MMM 'YY, eg: "May '19"')

% Get date range

% Properly display the organization of a work Experience

% Escape special characters in text
% Currently cleans: '$'


% List the contents of a nested list of lists with a delimiter

% Only print title if it exists


% Expand a list of objects, chaining 'hasName' together with comma delimiter

% Format a portfolio date (i.e. just the year)

\begin{document}
    % CV Header - Name, address, website and email
    %---------------------------------------------
    \centerline{ \color{\highlightcolor} \huge \textbf{Rukmal Weerawarana}}

    \vspace{.5em}

    {\color{\highlightcolor} \centerline{rukmal.weerawarana@gmail.com $\; \bullet \;$ https://rukmal.me $\; \bullet \;$ (206) 839-6891} }

    \noindent{\rule{\linewidth}{.2em}}


    % Education
    %---------------------------------------------

    
        \section{Education}

    % Iterating over degrees
    
        \begin{tabularx}{\tabularxwidth}{X r}
            % First line - degree title, and location
            \textbf{Master of Science (Financial Engineering)} & \textbf{
    Hoboken, 
        NJ} \\
            % Degree university, and location
            \quad \textcolor{\highlightcolor}{Stevens Institute of Technology} & 
    May ‘19 \\
            % Degree school and department
            \quad School of Business (Division of Financial Engineering) & \\
            % Degree GPA (if any)
            
                \quad \textit{GPA (4.0 Scale):} 3.8 & \\
            
            % Degree awards (if any)
            
                
                    
                        \quad \textit{Awards:} 
    Provost's Master's Fellowship & \\
                    
                
                    
                
                    
                
            
        \end{tabularx}

        % Add vertical space if not last iteration
        
            \vspace{.5em}
        

    
        \begin{tabularx}{\tabularxwidth}{X r}
            % First line - degree title, and location
            \textbf{Bachelor of Arts in Business Administration (Finance)} & \textbf{
    Seattle, 
        WA} \\
            % Degree university, and location
            \quad \textcolor{\highlightcolor}{University of Washington} & 
    Jun ‘17 \\
            % Degree school and department
            \quad Michael G. Foster School of Business (Department of Finance and Business Economics) & \\
            % Degree GPA (if any)
            
            % Degree awards (if any)
            
                
                    
                
            
        \end{tabularx}

        % Add vertical space if not last iteration
        
            \vspace{.5em}
        

    
        \begin{tabularx}{\tabularxwidth}{X r}
            % First line - degree title, and location
            \textbf{Certificate in Quantitative Fundamentals of Computational Finance} & \textbf{
    Seattle, 
        WA} \\
            % Degree university, and location
            \quad \textcolor{\highlightcolor}{University of Washington} & 
    Aug ‘16 \\
            % Degree school and department
            \quad College of Arts and Sciences (Department of Applied Mathematics) & \\
            % Degree GPA (if any)
            
            % Degree awards (if any)
            
        \end{tabularx}

        % Add vertical space if not last iteration
        

    

    % Work Experience
    %---------------------------------------------

    
        \section{Work Experience}

    
    % Put in minipage to prevent mid-entry pagebreak
    \begin{minipage}{\tabularxwidth}

        \begin{tabularx}{\tabularxwidth}{X r}
            % First line - job title and location
            \textbf{Research Collaborator} & \textbf{Troy, 
        NY} \\
            % Display employment organization row
            
    % Determining number of unique parent organizations
    
    
        % Last element of the list corresponds to the highest level (i.e. 'parent') orgNone
    

    
        % NOTE: This is for the case that there is no more than one base parent organization
        % Main employment organization and date range
        \textcolor{\highlightcolor}{Rensselaer Polytechnic Institute} & 
        
    Sep ‘21 - Present \\
        % Sub organizations (if any)
        
            % Top level sub organization once
            \textit{SCIENCE Blockchain Project;}
                \textit{Institute for Data Exploration and Applications (IDEA)} & \\
        
    
            % Other job titles
            
        \end{tabularx}

        % descriptions
        \begin{itemize}[noitemsep, topsep=3pt, parsep=0pt, partopsep=0pt]
            
                \item 
    Assist in the research and development process of a trustworthy, accountable data sharing ecosystem for biomedical research.
            
                \item 
    Lead efforts in developing cloud infrastructure to process and analyze data to build the underlying collaborator index. Currently exploring tokenomics policies for incentivizing data sharing in research networks.
            
                \item 
    Help mentor undergraduate students by providing assistance with research organization, infrastructure development, project implementation, and publication drafting.
            
                \item 
    The goal of the SCIENCE project is to design an expressive, provenance-centric language; SCIENCE Capability-based, Intention-centric, Experiment-oriented, Networked Collaborative Expression.
            
        \end{itemize}

        % Add vertical space if not last iteration
        
            \vspace{.5em}
        

    \end{minipage}
    
    % Put in minipage to prevent mid-entry pagebreak
    \begin{minipage}{\tabularxwidth}

        \begin{tabularx}{\tabularxwidth}{X r}
            % First line - job title and location
            \textbf{Software Engineer (Data Science)} & \textbf{Seattle, 
        WA} \\
            % Display employment organization row
            
    % Determining number of unique parent organizations
    
    
        % Last element of the list corresponds to the highest level (i.e. 'parent') orgNone
    

    
        % NOTE: This is for the case that there is no more than one base parent organization
        % Main employment organization and date range
        \textcolor{\highlightcolor}{ExtraHop Networks} & 
        
    Sep ‘19 - Present \\
        % Sub organizations (if any)
        
            % Top level sub organization once
            \textit{Unusual Behaviors Group;}
                \textit{Data Science R\&D} & \\
        
    
            % Other job titles
            
        \end{tabularx}

        % descriptions
        \begin{itemize}[noitemsep, topsep=3pt, parsep=0pt, partopsep=0pt]
            
                \item 
    Designed and developed high throughput data pipelines to transport and load semi structured data across multiple globally distributed data centers. Experienced in creating enriched data sets to deliver insights to internal and external stakeholders.
            
                \item 
    Produced numerous Data Science applications for both research and production use, with modern AWS serverless technologies and architectures. Applied modern Data Science algorithms and methodologies to Big Data to deliver insights to our customers.
            
                \item 
    Formalized, engineered, and managed the core internal ETL pipeline for over 2 years, delivering value to internal teams across the business. Scaled the core pipeline to handle an order of magnitude increase in data volume, while increasing functionality, due to increased adoption of ExtraHop's Cybersecurity product, Reveal(X) 360.
            
                \item 
    Enhanced ExtraHop's core threat hunting ability as a detector writer, applying unsupervised learning algorithms to Big Data scale traffic flows in the cloud. Implemented real-time detectors that act as early warning signals of potential bad actors on corporate networks across our customer base.
            
                \item 
    Participated in recruitment efforts, and interviewed teammates before, and during the COVID pandemic. Assisted with mentoring and on-boarding new team members.
            
                \item 
    ExtraHop secures hybrid cloud enterprises by providing a real-time stream processing sensor (both at the data center, and in the cloud), that transforms unstructured network packets to structured wire data at line rate. Data Science ideates, architects, deploys, and delivers a SaaS NDR solution to help secure the networks of our customers around the world.
            
                \item 
    ExtraHop Networks was acquired by Bain Capital and Crosspoint Capital Partners for \$900 million in July 2021.
            
        \end{itemize}

        % Add vertical space if not last iteration
        
            \vspace{.5em}
        

    \end{minipage}
    
    % Put in minipage to prevent mid-entry pagebreak
    \begin{minipage}{\tabularxwidth}

        \begin{tabularx}{\tabularxwidth}{X r}
            % First line - job title and location
            \textbf{Research Assistant} & \textbf{Hoboken, 
        NJ} \\
            % Display employment organization row
            
    % Determining number of unique parent organizations
    
    
        % Last element of the list corresponds to the highest level (i.e. 'parent') orgNone
    
        % Last element of the list corresponds to the highest level (i.e. 'parent') orgNone
    

    
        % NOTE: This is for the case that there is no more than one base parent organization
        % Main employment organization and date range
        \textcolor{\highlightcolor}{Stevens Institute of Technology} & 
        
    Aug ‘18 - 
    May ‘19 \\
        % Sub organizations (if any)
        
            % Top level sub organization once
            \textit{Sensorimotor Control Laboratory;}
                \textit{Stevens Institute for Artificial Intelligence} \textit{\&}
                \textit{Department of Biomedical Engineering} & \\
        
    
            % Other job titles
            
        \end{tabularx}

        % descriptions
        \begin{itemize}[noitemsep, topsep=3pt, parsep=0pt, partopsep=0pt]
            
                \item 
    Designed and implemented algorithms to assess and classify tremor severity in patients with late-stage Parkinson's Disease.
            
                \item 
    Created a highly scalable and extensible web application to be used by the researchers in the lab during this project. This web application incorporated HIPAA-compliant data storage and access, as well as efficient cluster management with Docker and Kubernetes.
            
                \item 
    Provided input to hardware research based on statistical tremor analysis results, with the goal of designing a complete low-cost system for tremor analysis. Contributed to additive manufacturing models and designing sensor arrays for a data collection glove to be used for the tremor analysis project.
            
                \item 
    Performed cost comparisons to identify the most economical Infrastructure-as-a-Service (IaaS) provider, given the storage, computation, and cost constraints of the project.
            
        \end{itemize}

        % Add vertical space if not last iteration
        
            \vspace{.5em}
        

    \end{minipage}
    
    % Put in minipage to prevent mid-entry pagebreak
    \begin{minipage}{\tabularxwidth}

        \begin{tabularx}{\tabularxwidth}{X r}
            % First line - job title and location
            \textbf{Summer Research Fellow} & \textbf{Troy, 
        NY} \\
            % Display employment organization row
            
    % Determining number of unique parent organizations
    
    
        % Last element of the list corresponds to the highest level (i.e. 'parent') orgNone
    
        % Last element of the list corresponds to the highest level (i.e. 'parent') orgNone
    

    
        % NOTE: This is for the case that there is more than one base parent organization
        % Main employment organization and date range
        \textcolor{\highlightcolor}{RPI-IBM HEALS Research Center} & 
        
    May ‘18 - 
    Aug ‘18 \\
        % Parent organizations (if any)
        
            
                \textit{AI Horizons Network; IBM Research} & \\
            
                \textit{Tetherless World Constellation; Rensselaer Polytechnic Institute} & \\
            
        
    
            % Other job titles
            
        \end{tabularx}

        % descriptions
        \begin{itemize}[noitemsep, topsep=3pt, parsep=0pt, partopsep=0pt]
            
                \item 
    Led the design and development of the PaperRank Framework, a methodology for deriving probabilistic community trust in academic publications. PaperRank utilized the PageRank algorithm, coupled with a Gamma Mixture Model applied to citation networks of academic publications. A proof-of-concept was implemented, from extraction to final trust score computation, analyzing over 14 Million articles from the NCBI PubMed Database.
            
                \item 
    Formulated and implemented novel strategies for semantically-enhanced automated extraction of medical directives from Clinical Practice Guidelines (CPGs), for eventual inclusion in a knowledge graph of Diabetes diagnosis and treatment directives. Built the 'Guideline Explorer', a tool for efficiently visualizing and examining the American Diabetes Association's 2018 CPGs.
            
                \item 
    Explored the field of 'Semantalytics', which lies at the intersection of Semantics and Analytics. Drafted a Vision statement for the future exploration of this novel field of research, through the lens of bioinformatics.
            
                \item 
    Formulated an Electronic Health Record (EHR) simulation engine, which utilized Monte Carlo simulations based on a generalized population heuristic to vary idiosyncratic patient attributes. The EHR Simulation engine would suggest medical tests that would be statistically likely to identify previously unknown medical issues.
            
                \item 
    Developed the 'Guideline Analysis Framework', a mathematical formulation of CPGs. This framework was designed to enable the comparison of various CPGs from differing medical authorities addressing the same set of diseases, and to detect disparities in treatment directives.
            
        \end{itemize}

        % Add vertical space if not last iteration
        
            \vspace{.5em}
        

    \end{minipage}
    
    % Put in minipage to prevent mid-entry pagebreak
    \begin{minipage}{\tabularxwidth}

        \begin{tabularx}{\tabularxwidth}{X r}
            % First line - job title and location
            \textbf{Laboratory Assistant} & \textbf{Hoboken, 
        NJ} \\
            % Display employment organization row
            
    % Determining number of unique parent organizations
    
    
        % Last element of the list corresponds to the highest level (i.e. 'parent') orgNone
    
        % Last element of the list corresponds to the highest level (i.e. 'parent') orgNone
    

    
        % NOTE: This is for the case that there is no more than one base parent organization
        % Main employment organization and date range
        \textcolor{\highlightcolor}{Stevens Institute of Technology} & 
        
    Sep ‘17 - 
    Dec ‘18 \\
        % Sub organizations (if any)
        
            % Top level sub organization once
            \textit{Hanlon Financial Systems Laboratory;}
                \textit{Stevens Institute for Artificial Intelligence} \textit{\&}
                \textit{School of Business} & \\
        
    
            % Other job titles
            
        \end{tabularx}

        % descriptions
        \begin{itemize}[noitemsep, topsep=3pt, parsep=0pt, partopsep=0pt]
            
                \item 
    Spearheaded an effort to discover and implement new processes, adopt more adaptable technology, and increase functional collaboration, to help realize the teaching and research goals of the lab.
            
                \item 
    Assisted in student project guidance, and extra-curricular instruction. Introduced interactive programming technology to aid in the Introduction to C++ course taught to MSFE students.
            
                \item 
    Facilitated the daily operations of the lab, including assisting instructors and students (Graduate and Undergraduate), and maintaining hardware and software resources.
            
        \end{itemize}

        % Add vertical space if not last iteration
        
            \vspace{.5em}
        

    \end{minipage}
    
    % Put in minipage to prevent mid-entry pagebreak
    \begin{minipage}{\tabularxwidth}

        \begin{tabularx}{\tabularxwidth}{X r}
            % First line - job title and location
            \textbf{Business Management Team Lead} & \textbf{Seattle, 
        WA} \\
            % Display employment organization row
            
    % Determining number of unique parent organizations
    
    
        % Last element of the list corresponds to the highest level (i.e. 'parent') orgNone
    

    
        % NOTE: This is for the case that there is no more than one base parent organization
        % Main employment organization and date range
        \textcolor{\highlightcolor}{University of Washington} & 
        
    May ‘16 - 
    Aug ‘17 \\
        % Sub organizations (if any)
        
            % Top level sub organization once
            \textit{UW Hyperloop;}
                \textit{College of Engineering} & \\
        
    
            % Other job titles
            
                Other Titles: Impact Development Team Lead, Control Systems Team Member & \\
            
        \end{tabularx}

        % descriptions
        \begin{itemize}[noitemsep, topsep=3pt, parsep=0pt, partopsep=0pt]
            
                \item 
    Represented the University of Washington at the inaugural SpaceX Hyperloop Pod Competition in Hawthorne, CA. We placed 4th in the United States, and 6th globally; the competition initially received 1,700 team proposals, which were narrowed down to 30 finalists.
            
                \item 
    Led the Business Management Team to launch a highly successful crowdfunding campaign, raising over \$20,000 in cash (with an initial goal of \$10,000), and over \$80,000 of source materials used in the construction of the Pod. The collective effort of the team led us to have the lowest-cost Pod among the 30 final teams.
            
                \item 
    Spearheaded the sourcing and delivery of over \$50,000 of raw material, including high-density Carbon Fiber, release agents, and powerful Neodymium magnets for the final Pod assembly.
            
                \item 
    Led sponsor outreach, and maintained relationships with internal (at the University of Washington) and external supervisors and supporters.
            
                \item 
    Assisted the Controls Team Lead in the final design and implementation of electronic mapping and wiring on the Pod.
            
                \item 
    Contributed to late-stage troubleshooting efforts that led to the successful implementation and deployment of Halbach arrays on the Pod, which facilitated levitation and magnetic propulsion on the test track.
            
                \item 
    Explored the transformative economic and social effects a hypothetical Hyperloop system could have on the Pacific Northwest of the United States.
            
        \end{itemize}

        % Add vertical space if not last iteration
        
            \vspace{.5em}
        

    \end{minipage}
    
    % Put in minipage to prevent mid-entry pagebreak
    \begin{minipage}{\tabularxwidth}

        \begin{tabularx}{\tabularxwidth}{X r}
            % First line - job title and location
            \textbf{Undergraduate Research Assistant} & \textbf{Seattle, 
        WA} \\
            % Display employment organization row
            
    % Determining number of unique parent organizations
    
    
        % Last element of the list corresponds to the highest level (i.e. 'parent') orgNone
    

    
        % NOTE: This is for the case that there is no more than one base parent organization
        % Main employment organization and date range
        \textcolor{\highlightcolor}{University of Washington} & 
        
    Mar ‘15 - 
    Jul ‘15 \\
        % Sub organizations (if any)
        
            % Top level sub organization once
            \textit{UW CubeSAT Team;}
                \textit{Advanced Propulsion Laboratory} & \\
        
    
            % Other job titles
            
                Other Titles: Avionics Team Lead & \\
            
        \end{tabularx}

        % descriptions
        \begin{itemize}[noitemsep, topsep=3pt, parsep=0pt, partopsep=0pt]
            
                \item 
    Led the avionics team to design a communications software architecture paradigm for the on-board computer systems of the (previously; changed to LEO by NASA) Lunar-orbit CubeSAT.
            
                \item 
    Investigated the effect of radiation beyond the Van-Allen belts (experienced during translunar flight). Recommended ideal processor architectures and other redundant measures that may be taken to mitigate the effect of this radiation on the flight systems of the CubeSAT.
            
                \item 
    Proposed a system of Hamming codes to increase data transfer fidelity during data dumps from the CubeSAT to the ground station at the University of Washington. Identified specific data encoding paradigms to increase data throughput to the Earth downlink from the CubeSAT.
            
                \item 
    Presented a final communications software architecture proposal to a panel of researchers from the Advanced Propulsion Laboratory and NASA.
            
                \item 
    The UW CubeSAT was deployed to low-Earth orbit as secondary payload on a Cygnus cargo spacecraft, which launched aboard a Northup Grumman Antares rocket in 2019.
            
        \end{itemize}

        % Add vertical space if not last iteration
        
            \vspace{.5em}
        

    \end{minipage}
    
    % Put in minipage to prevent mid-entry pagebreak
    \begin{minipage}{\tabularxwidth}

        \begin{tabularx}{\tabularxwidth}{X r}
            % First line - job title and location
            \textbf{Software Engineering Team Lead} & \textbf{Seattle, 
        WA} \\
            % Display employment organization row
            
    % Determining number of unique parent organizations
    
    
        % Last element of the list corresponds to the highest level (i.e. 'parent') orgNone
    

    
        % NOTE: This is for the case that there is no more than one base parent organization
        % Main employment organization and date range
        \textcolor{\highlightcolor}{ZocialGPA, Inc.} & 
        
    Feb ‘15 - 
    Jan ‘16 \\
        % Sub organizations (if any)
        
    
            % Other job titles
            
                Other Titles: Software Engineering Intern & \\
            
        \end{tabularx}

        % descriptions
        \begin{itemize}[noitemsep, topsep=3pt, parsep=0pt, partopsep=0pt]
            
                \item 
    Designed a highly scalable and efficient software ETL stack, building datasets from various social networking platforms, including Facebook, Twitter, and LinkedIn.
            
                \item 
    Developed natural language processing and sentiment analysis algorithms to derive "social GPA" scores from a user's social profiles.
            
                \item 
    Refactored and modularized the entire company codebase, to enable efficient component-based auto scaling with Apache Stratos and Amazon AWS.
            
                \item 
    Implemented a mobile-first web end user interface for the platform, and redesigned the internal company management console.
            
        \end{itemize}

        % Add vertical space if not last iteration
        
            \vspace{.5em}
        

    \end{minipage}
    
    % Put in minipage to prevent mid-entry pagebreak
    \begin{minipage}{\tabularxwidth}

        \begin{tabularx}{\tabularxwidth}{X r}
            % First line - job title and location
            \textbf{Software Engineering Intern} & \textbf{Colombo, 
        Sri Lanka} \\
            % Display employment organization row
            
    % Determining number of unique parent organizations
    
    
        % Last element of the list corresponds to the highest level (i.e. 'parent') orgNone
    

    
        % NOTE: This is for the case that there is no more than one base parent organization
        % Main employment organization and date range
        \textcolor{\highlightcolor}{WSO2, Inc.} & 
        
    Jun ‘14 - 
    Sep ‘14 \\
        % Sub organizations (if any)
        
            % Top level sub organization once
            \textit{Apache Stratos Team;}
                \textit{} & \\
        
    
            % Other job titles
            
        \end{tabularx}

        % descriptions
        \begin{itemize}[noitemsep, topsep=3pt, parsep=0pt, partopsep=0pt]
            
                \item 
    Investigated the viability of alternate hypervisor stacks for eventual integration with the Apache Stratos Platform-as-a-Service (PaaS) framework.
            
                \item 
    Developed a new user interface for the Stratos Manager Console using the JaggeryJS MVC framework, which was packaged and shipped with Stratos version 4.1.0.
            
                \item 
    Conducted isolated integration tests with the CoreOS+Docker (LXC-based) hypervisor stack as an alternative to the existing hypervisor (KVM) used in Stratos.
            
        \end{itemize}

        % Add vertical space if not last iteration
        
            \vspace{.5em}
        

    \end{minipage}
    
    % Put in minipage to prevent mid-entry pagebreak
    \begin{minipage}{\tabularxwidth}

        \begin{tabularx}{\tabularxwidth}{X r}
            % First line - job title and location
            \textbf{Undergraduate Research Assistant} & \textbf{Seattle, 
        WA} \\
            % Display employment organization row
            
    % Determining number of unique parent organizations
    
    
        % Last element of the list corresponds to the highest level (i.e. 'parent') orgNone
    

    
        % NOTE: This is for the case that there is no more than one base parent organization
        % Main employment organization and date range
        \textcolor{\highlightcolor}{University of Washington} & 
        
    Apr ‘14 - 
    Aug ‘14 \\
        % Sub organizations (if any)
        
            % Top level sub organization once
            \textit{Mullins Molecular Retrovirology Laboratory;}
                \textit{Department of Microbiology} & \\
        
    
            % Other job titles
            
        \end{tabularx}

        % descriptions
        \begin{itemize}[noitemsep, topsep=3pt, parsep=0pt, partopsep=0pt]
            
                \item 
    Assisted the Lead Scientific Programmer of the lab in identifying and extracting sequences of mutated genes in human genome sequences.
            
                \item 
    Developed algorithms which aided in the creation of targeted retroviral therapies for patients with HIV/AIDS.
            
                \item 
    Created efficient and scalable data traversal algorithms for the ingestion and transformation of large human genome sequences.
            
                \item 
    Utilized Hyperfreq, a tool for Bayesian analysis of APOBEC3G-induced hypermutations, to identify sequences of the genome likely to have been mutated.
            
        \end{itemize}

        % Add vertical space if not last iteration
        

    \end{minipage}
    

    % Publications
    %---------------------------------------------

    
        \section{Publications}

    
        \begin{minipage}{\tabularxwidth}
        \begin{tabularx}{\tabularxwidth}{X}
            % Nested tabular to handle title and date in same column
            {
                \begin{tabularx}{\tabularxwidth}{@{}X r}
                    % Prefix with the current status (if any)
                    % Publication title
                    \textbf{Learned Sectors: A fundamentals-driven sector reclassification project} &
                    % Date on the right
                    \textbf{
        2019} \\
                \end{tabularx}
            } \\
            % Authors
            Rukmal Weerawarana, Yiyi Zhu, Yuzhen He \\

            % TODO: This part can be improved in the future
            % NOTE: Assuming that both publication & conference/journal exists if one does
            
                \textit{arXiv preprint; arXiv:1906.03935} \\
            
            % Printing website (if exists)
            
                \url{https://arxiv.org/abs/1906.03935} \\
            
            % Printing DOI (if exists)
            
        \end{tabularx}

        % descriptions
        \begin{itemize}[noitemsep, topsep=3pt, parsep=0pt, partopsep=0pt]
            
                \item 
    Market sectors play a key role in enabling the efficient flow of capital through the modern Global economy. An analysis of existing sectorization heuristics show that they are not entirely quantitatively driven, but rather are highly subjective and rooted in dogma. To this end, we introduce a new fundamentals-driven Learned Sectors heuristic.
            
                \item 
    Using the HCA heuristic, we generate a set of 60 potential candidate learned sector universes. We then introduce reIndexer, a backtest-driven sector universe evaluation research tool, to rank the candidate sector universes produced by our learned sector classification heuristic.
            
                \item 
    This rank was utilized to identify the risk-adjusted return optimal learned sector universe as being the universe generated under CLINK (i.e. complete linkage), with 17 sectors. The optimal learned sector universe was tested against the benchmark GICS classification universe with reIndexer, outperforming on both absolute portfolio value, and risk-adjusted return over the backtest period.
            
        \end{itemize}

        % Add vertical space if not last iteration
        
            \vspace{.5em}
        

        \end{minipage}
    
        \begin{minipage}{\tabularxwidth}
        \begin{tabularx}{\tabularxwidth}{X}
            % Nested tabular to handle title and date in same column
            {
                \begin{tabularx}{\tabularxwidth}{@{}X r}
                    % Prefix with the current status (if any)
                        \textit{(Draft) }
                    % Publication title
                    \textbf{Inferring Community Trust from Citation Graphs} &
                    % Date on the right
                    \textbf{
        2019} \\
                \end{tabularx}
            } \\
            % Authors
            Jamie McCusker, Rukmal Weerawarana, Alexander New, Kristin P. Bennett, Deborah L. McGuinness \\

            % TODO: This part can be improved in the future
            % NOTE: Assuming that both publication & conference/journal exists if one does
            
            % Printing website (if exists)
            
                \url{https://drive.google.com/open?id=1SlSfZrwOQYP0mrKbrGLjjAFzWCk-qlwm} \\
            
            % Printing DOI (if exists)
            
        \end{tabularx}

        % descriptions
        \begin{itemize}[noitemsep, topsep=3pt, parsep=0pt, partopsep=0pt]
            
                \item 
    We introduce the PaperRank scoring algorithm; a proxy of scientific community trust in a given publication. This score is derived from the classic PageRank algorithm (applied to academic citation networks), in conjunction with a one-dimensional Gamma Mixture Model to normalize the PageRank scores on a 3-group publication notoriety heuristic.
            
                \item 
    The key contributions of this publication are the PaperRank Framework (currently configured for use with the NCBI PubMed Database), and a generalized algorithm for optimizing a Gamma Mixture Model (GMM).
            
                \item 
    A key application of PaperRank would be to underwrite provenance-enabled Knowledge Graphs (where assertions are justified by academic publication references) with a 'trust' heuristic, to enable quasi-probabilistic behavior during inference and traversal.
            
        \end{itemize}

        % Add vertical space if not last iteration
        
            \vspace{.5em}
        

        \end{minipage}
    
        \begin{minipage}{\tabularxwidth}
        \begin{tabularx}{\tabularxwidth}{X}
            % Nested tabular to handle title and date in same column
            {
                \begin{tabularx}{\tabularxwidth}{@{}X r}
                    % Prefix with the current status (if any)
                    % Publication title
                    \textbf{Semantic Modeling of Cohort Descriptions in Research Studies} &
                    % Date on the right
                    \textbf{
        2018} \\
                \end{tabularx}
            } \\
            % Authors
            Shruthi Chari, Rukmal Weerawarana, Oshani Seneviratne, Jamie McCusker, Deborah L. McGuinness, Amar Das \\

            % TODO: This part can be improved in the future
            % NOTE: Assuming that both publication & conference/journal exists if one does
            
                \textit{Knowledge Representation and Semantics Workshop; AMIA 2018 Annual Symposium} \\
            
            % Printing website (if exists)
            
                \url{https://tw.rpi.edu/web/doc/semantic_modeling_of_cohort} \\
            
            % Printing DOI (if exists)
            
        \end{tabularx}

        % descriptions
        \begin{itemize}[noitemsep, topsep=3pt, parsep=0pt, partopsep=0pt]
            
                \item 
    This research addresses a key challenge faced by physicians using Clinical Practice Guideline recommendations; determining how well idiosyncratic cohort evidence generalizes to the greater clinical population.
            
                \item 
    The end goal of this system is to enable the parsing of publications broadly identified as research studies to extract cohort variables and exposure/intervention groups defined within the structured population descriptions to better inform treatment decisions.
            
        \end{itemize}

        % Add vertical space if not last iteration
        
            \vspace{.5em}
        

        \end{minipage}
    
        \begin{minipage}{\tabularxwidth}
        \begin{tabularx}{\tabularxwidth}{X}
            % Nested tabular to handle title and date in same column
            {
                \begin{tabularx}{\tabularxwidth}{@{}X r}
                    % Prefix with the current status (if any)
                    % Publication title
                    \textbf{What is a Knowledge Graph?} &
                    % Date on the right
                    \textbf{
        2018} \\
                \end{tabularx}
            } \\
            % Authors
            Jamie McCusker, John S. Erickson, Katherine Chastain, Sabbir Rashid, Rukmal Weerawarana, Marcello Bax, Deborah L. McGuinness \\

            % TODO: This part can be improved in the future
            % NOTE: Assuming that both publication & conference/journal exists if one does
            
            % Printing website (if exists)
            
                \url{https://drive.google.com/open?id=19uND_fkRTd_m-i-SYBznq1wuCq1IvHhO} \\
            
            % Printing DOI (if exists)
            
        \end{tabularx}

        % descriptions
        \begin{itemize}[noitemsep, topsep=3pt, parsep=0pt, partopsep=0pt]
            
                \item 
    This work attempts to synthesize a clear and unambiguous definition of a 'Knowledge Graph' that conforms to current knowledge graph research, while constraining the research space that may be considered a knowledge graph.
            
                \item 
    We evaluate a wide variety of knowledge resources, graphs, and other ontologies to determine if they qualify under our definition, and built a 'Knowledge Graph Catalog' to support this effort.
            
        \end{itemize}

        % Add vertical space if not last iteration
        

        \end{minipage}
    

    % Talks
    %---------------------------------------------

    
        \section{Talks}

    
        \begin{minipage}{\tabularxwidth}
        \begin{tabularx}{\tabularxwidth}{X}
            % Nested tabular to handle title and date in same column
            {
                \begin{tabularx}{\tabularxwidth}{@{}X r}
                    % Talk title
                    \textbf{Neural Ordinary Differential Equations} &
                    % Date on the right
                    \textbf{
        2020} \\
                \end{tabularx}
            } \\

            % Collaborators
            

            % Printing website (if exists)
            
                \url{https://drive.google.com/file/d/1fqVH6GJe1TcRyD6tL2cDUXkkkhyRq4vY} \\
            
            % Printing DOI (if exists)
            

            % description (only using one)
            
    A literature review of Neural Ordinary Differential Equations by Chen et al., a new family of deep neural network models that parameterizes the hidden state of a neural network.
        \end{tabularx}
        % Add vertical space if not last iteration
        
            \vspace{.5em}
        

        \end{minipage}
    
        \begin{minipage}{\tabularxwidth}
        \begin{tabularx}{\tabularxwidth}{X}
            % Nested tabular to handle title and date in same column
            {
                \begin{tabularx}{\tabularxwidth}{@{}X r}
                    % Talk title
                    \textbf{Learned Sectors} &
                    % Date on the right
                    \textbf{
        2019} \\
                \end{tabularx}
            } \\

            % Collaborators
            
                \textit{Collaborators: Yiyi Zhu, Yuzhen He} \\
            

            % Printing website (if exists)
            
                \url{https://drive.google.com/file/d/12DcqxSJAR-PTLZcRqZ6SAapKzQFBuwA4} \\
            
            % Printing DOI (if exists)
            

            % description (only using one)
            
    Learned Sectors project overview, covering the trained hierarchical clustering model, the reIndexer validation system, and a discussion of the final risk-adjusted return optimal sector universe.
        \end{tabularx}
        % Add vertical space if not last iteration
        
            \vspace{.5em}
        

        \end{minipage}
    
        \begin{minipage}{\tabularxwidth}
        \begin{tabularx}{\tabularxwidth}{X}
            % Nested tabular to handle title and date in same column
            {
                \begin{tabularx}{\tabularxwidth}{@{}X r}
                    % Talk title
                    \textbf{Knowledge Graph Fundamentals} &
                    % Date on the right
                    \textbf{
        2018} \\
                \end{tabularx}
            } \\

            % Collaborators
            

            % Printing website (if exists)
            
                \url{https://drive.google.com/file/d/1f21S_QZtm6aYiYLnXPA5opgsoBdIX7zX} \\
            
            % Printing DOI (if exists)
            

            % description (only using one)
            
    An overview of the fundamental technology powering modern knowledge graphs, focusing on the concepts of semantic data, ontologies, and inference.
        \end{tabularx}
        % Add vertical space if not last iteration
        
            \vspace{.5em}
        

        \end{minipage}
    
        \begin{minipage}{\tabularxwidth}
        \begin{tabularx}{\tabularxwidth}{X}
            % Nested tabular to handle title and date in same column
            {
                \begin{tabularx}{\tabularxwidth}{@{}X r}
                    % Talk title
                    \textbf{High Frequency Trading (HFT) - A Deep Dive} &
                    % Date on the right
                    \textbf{
        2017} \\
                \end{tabularx}
            } \\

            % Collaborators
            

            % Printing website (if exists)
            
                \url{https://drive.google.com/file/d/1I7JuZhVzsAT84xe88lLsbHbSIa-Ev7eF} \\
            
            % Printing DOI (if exists)
            

            % description (only using one)
            
    A deep dive into High Frequency Trading (HFT), covering market microstructure, exchange dynamics, regulatory implications, electronic order execution models, RegNMS, algorithmic trading, and popular HFT-driven strategies for exploiting arbitrage opportunities.
        \end{tabularx}
        % Add vertical space if not last iteration
        
            \vspace{.5em}
        

        \end{minipage}
    
        \begin{minipage}{\tabularxwidth}
        \begin{tabularx}{\tabularxwidth}{X}
            % Nested tabular to handle title and date in same column
            {
                \begin{tabularx}{\tabularxwidth}{@{}X r}
                    % Talk title
                    \textbf{Leap into the Future with Leap Motion} &
                    % Date on the right
                    \textbf{
        2014} \\
                \end{tabularx}
            } \\

            % Collaborators
            

            % Printing website (if exists)
            
                \url{http://uwhackers.github.io/leap-motion-slides/} \\
            
            % Printing DOI (if exists)
            

            % description (only using one)
            
    An introduction and overview of the Leap Motion; the underlying technology, JavaScript API, and a WebSocket game demo.
        \end{tabularx}
        % Add vertical space if not last iteration
        

        \end{minipage}
    

    % Knowledge Areas
    %---------------------------------------------

    
        \section{Selected Knowledge Areas}

    
        % iterating over knowledge areas
        \begin{tabularx}{\tabularxwidth}{X}
            % first line - knowldge area name in bold
            \textbf{Artificial Intelligence and Data Science} \\
            % listing all constituent subjects
            
    
            Anomaly Detection, 
            Bibliometrics, 
            Cluster Analysis, 
            Data Visualization, 
            Dimensionality Reduction, 
            Graph Analytics, 
            Knowledge Representation, 
            Machine Learning, 
            Natural Language Processing, 
            Predictive Modeling, 
            Semantic Analysis, 
            Semi-Supervised Learning, 
            Statistical Classification, 
            Time Series Analysis, 
            Unsupervised Learning \\
        \end{tabularx}

        % Add vertical space if not last iteration
        
            \vspace{.5em}
        

    
        % iterating over knowledge areas
        \begin{tabularx}{\tabularxwidth}{X}
            % first line - knowldge area name in bold
            \textbf{Computer Science and Software Engineering} \\
            % listing all constituent subjects
            
    
            Algorithms, 
            Cloud Computing, 
            Data Structures, 
            Distributed Systems, 
            Object Oriented Design, 
            Operating Systems, 
            Rapid Prototyping, 
            Scalable System Architecture and Design, 
            Scientific Computing, 
            Unit Testing \\
        \end{tabularx}

        % Add vertical space if not last iteration
        
            \vspace{.5em}
        

    
        % iterating over knowledge areas
        \begin{tabularx}{\tabularxwidth}{X}
            % first line - knowldge area name in bold
            \textbf{Economics and Econometrics} \\
            % listing all constituent subjects
            
    
            Computational Econometrics, 
            Credit Risk Modeling, 
            Macroeconomics, 
            Managerial Economics, 
            Microeconomics, 
            Yield Curve Modeling \\
        \end{tabularx}

        % Add vertical space if not last iteration
        
            \vspace{.5em}
        

    
        % iterating over knowledge areas
        \begin{tabularx}{\tabularxwidth}{X}
            % first line - knowldge area name in bold
            \textbf{Finance} \\
            % listing all constituent subjects
            
    
            Advanced Derivatives, 
            Asset-Backed Securities, 
            Banking and Financial Systems, 
            Capital Budgeting, 
            Computational Finance, 
            Corporate Finance, 
            Exotic Derivative Pricing, 
            Financial Reporting, 
            Fixed Income, 
            Foreign Exchange Risk, 
            International Finance, 
            Managerial Accounting, 
            Market Microstructure, 
            Modern Portfolio Theory, 
            Risk Analytics \\
        \end{tabularx}

        % Add vertical space if not last iteration
        
            \vspace{.5em}
        

    
        % iterating over knowledge areas
        \begin{tabularx}{\tabularxwidth}{X}
            % first line - knowldge area name in bold
            \textbf{Mathematics} \\
            % listing all constituent subjects
            
    
            Advanced Probability Theory, 
            Combinatorics, 
            Convex Optimization, 
            Differential Equations, 
            Graph Theory, 
            Linear Algebra, 
            Mathematical Logic, 
            Multivariable Calculus, 
            Non-Convex Optimization, 
            Nonparametric Statistics, 
            Partial Differential Equations, 
            Real Analysis, 
            Statistical Modeling, 
            Stochastic Calculus \\
        \end{tabularx}

        % Add vertical space if not last iteration
        
            \vspace{.5em}
        

    
        % iterating over knowledge areas
        \begin{tabularx}{\tabularxwidth}{X}
            % first line - knowldge area name in bold
            \textbf{Other} \\
            % listing all constituent subjects
            
    
            AI Ethics, 
            Elementary Computational Genomics, 
            Elementary Quantum Computing, 
            Sociology of Science \\
        \end{tabularx}

        % Add vertical space if not last iteration
        

    


    % Skills and Technologies
    %---------------------------------------------

    
        \section{Skills and Technologies}

    
        % iterating over knowledge areas
        \begin{tabularx}{\tabularxwidth}{X}
            % first line - knowldge area name in bold
            \textbf{Application Software} \\
            % listing all constituent subjects
            
    
            Adobe Creative Cloud, 
            Autodesk AutoCAD, 
            Blender, 
            Bloomberg Terminal, 
            Microsoft Office, 
            Protege, 
            SolidWorks \\
        \end{tabularx}

        % Add vertical space if not last iteration
        
            \vspace{.5em}
        

    
        % iterating over knowledge areas
        \begin{tabularx}{\tabularxwidth}{X}
            % first line - knowldge area name in bold
            \textbf{Databases} \\
            % listing all constituent subjects
            
    
            Apache Jena, 
            Blazegraph, 
            MongoDB, 
            MySQL, 
            Neo4j, 
            Redis \\
        \end{tabularx}

        % Add vertical space if not last iteration
        
            \vspace{.5em}
        

    
        % iterating over knowledge areas
        \begin{tabularx}{\tabularxwidth}{X}
            % first line - knowldge area name in bold
            \textbf{Deployment, Orchestration, and Continuous Integration Tools} \\
            % listing all constituent subjects
            
    
            Amazon Web Services, 
            Docker, 
            GNU Make, 
            Google Cloud Platform, 
            Heroku, 
            Kubernetes, 
            Microsoft Azure, 
            Travis CI \\
        \end{tabularx}

        % Add vertical space if not last iteration
        
            \vspace{.5em}
        

    
        % iterating over knowledge areas
        \begin{tabularx}{\tabularxwidth}{X}
            % first line - knowldge area name in bold
            \textbf{Frameworks and Libraries} \\
            % listing all constituent subjects
            
    
            Electron, 
            Flask, 
            Go Revel, 
            NLTK, 
            Node.js, 
            R Shiny, 
            RDFLib, 
            SciPy, 
            Socket.IO, 
            Spring, 
            TensorFlow, 
            scikit-learn \\
        \end{tabularx}

        % Add vertical space if not last iteration
        
            \vspace{.5em}
        

    
        % iterating over knowledge areas
        \begin{tabularx}{\tabularxwidth}{X}
            % first line - knowldge area name in bold
            \textbf{Operating Systems} \\
            % listing all constituent subjects
            
    
            Linux (Ubuntu, Fedora, etc.), 
            Windows, 
            macOS \\
        \end{tabularx}

        % Add vertical space if not last iteration
        
            \vspace{.5em}
        

    
        % iterating over knowledge areas
        \begin{tabularx}{\tabularxwidth}{X}
            % first line - knowldge area name in bold
            \textbf{Programming and Scripting Languages} \\
            % listing all constituent subjects
            
    
            Bash, 
            CSS, 
            Go, 
            HTML, 
            Java, 
            JavaScript, 
            Jinja, 
            LaTeX, 
            MATLAB, 
            Perl, 
            Python, 
            R, 
            SPARQL, 
            SQL \\
        \end{tabularx}

        % Add vertical space if not last iteration
        
            \vspace{.5em}
        

    
        % iterating over knowledge areas
        \begin{tabularx}{\tabularxwidth}{X}
            % first line - knowldge area name in bold
            \textbf{Reproducible Research Tools} \\
            % listing all constituent subjects
            
    
            GitHub, 
            Google Colaboratory, 
            Knitr, 
            Microsoft Azure Notebooks, 
            Overleaf, 
            Project Jupyter, 
            Read the Docs \\
        \end{tabularx}

        % Add vertical space if not last iteration
        
            \vspace{.5em}
        

    
        % iterating over knowledge areas
        \begin{tabularx}{\tabularxwidth}{X}
            % first line - knowldge area name in bold
            \textbf{Soft Skills} \\
            % listing all constituent subjects
            
    
            Conflict Resolution, 
            Excellent Communication Skills, 
            Excellent Writing Skills, 
            Extensive Leadership Experience, 
            Project Management, 
            Public Speaking \\
        \end{tabularx}

        % Add vertical space if not last iteration
        

    


    % Projects
    %---------------------------------------------

    
        \section{Selected Projects}

    

        \begin{tabularx}{\tabularxwidth}{X}
            % Nested tabular to handle title (fancy, with skills) and date in same row
                {
                    \begin{tabularx}{\tabularxwidth}{@{}X r}
                        % First line - project name in bold, and skills/tech in brackets
                        \textbf{Precis}
                            (Jinja, LaTeX, Python, SPARQL)
                        % Alignment separator (after if so it get triggered either way)
                        &
                        % Date on the right
                        \textbf{
        2021} \\
                    \end{tabularx}
                } \\

            % Collaborators
            

            % Awards
            

            % Printing website (if exists)
            
                \url{https://precis.rukmal.me} \\
            

            % description (only using one)
            
    Precis is an Ontology for modeling personal professional metadata. The extended Precis toolkit also includes a Pythonic search API for the Ontology, a JSON data loader, and an extensible templating engine. \\

        \end{tabularx}

        % Add vertical space if not last iteration
        
            \vspace{.5em}
        

    

        \begin{tabularx}{\tabularxwidth}{X}
            % Nested tabular to handle title (fancy, with skills) and date in same row
                {
                    \begin{tabularx}{\tabularxwidth}{@{}X r}
                        % First line - project name in bold, and skills/tech in brackets
                        \textbf{fe621}
                            (LaTeX, Python, scikit-learn)
                        % Alignment separator (after if so it get triggered either way)
                        &
                        % Date on the right
                        \textbf{
        2019} \\
                    \end{tabularx}
                } \\

            % Collaborators
            

            % Awards
            

            % Printing website (if exists)
            
                \url{https://git.rukmal.me/FE-621-Homework} \\
            

            % description (only using one)
            
    fe621 is a Python library that provides functionality for lattice based derivative pricing models, exotic option picing, Monte Carlo simulations, numerical differentiation and integration, and optimization. \\

        \end{tabularx}

        % Add vertical space if not last iteration
        
            \vspace{.5em}
        

    

        \begin{tabularx}{\tabularxwidth}{X}
            % Nested tabular to handle title (fancy, with skills) and date in same row
                {
                    \begin{tabularx}{\tabularxwidth}{@{}X r}
                        % First line - project name in bold, and skills/tech in brackets
                        \textbf{reIndexer}
                            (Python, SciPy)
                        % Alignment separator (after if so it get triggered either way)
                        &
                        % Date on the right
                        \textbf{
        2019} \\
                    \end{tabularx}
                } \\

            % Collaborators
            

            % Awards
            

            % Printing website (if exists)
            
                \url{https://git.rukmal.me/reIndexer} \\
            

            % description (only using one)
            
    reIndexer is a research tool for the backtest-driven evaluation of different sectorization heuristics, using a system of synthetic ETFs, and efficient portfolios of those synthetic ETFs. \\

        \end{tabularx}

        % Add vertical space if not last iteration
        
            \vspace{.5em}
        

    

        \begin{tabularx}{\tabularxwidth}{X}
            % Nested tabular to handle title (fancy, with skills) and date in same row
                {
                    \begin{tabularx}{\tabularxwidth}{@{}X r}
                        % First line - project name in bold, and skills/tech in brackets
                        \textbf{HTKG and NYPD-Compstat-LD}
                            (Kubernetes, Neo4j, Python)
                        % Alignment separator (after if so it get triggered either way)
                        &
                        % Date on the right
                        \textbf{
        2019} \\
                    \end{tabularx}
                } \\

            % Collaborators
            
                \textit{Collaborators:} Ryan Hartman, Ayush Kalla, Kovid Shukla, Sanket Saharkar \\
            

            % Awards
            
                    \textit{Awards:}
                First Place [Technology] (IBM Bluehack Against Human Trafficking)

            % Printing website (if exists)
            
                \url{https://git.rukmal.me/NYPD-Compstat-LD} \\
            

            % description (only using one)
            
    HTKG (Human Trafficking Knolwedge Graph), and NYPD-Compstat-LD (the main data ingestion engine) is a knowledge graph platform for linking suspected human trafficking advertisements with crime data from the NYPD to retroactively assess trends. \\

        \end{tabularx}

        % Add vertical space if not last iteration
        
            \vspace{.5em}
        

    

        \begin{tabularx}{\tabularxwidth}{X}
            % Nested tabular to handle title (fancy, with skills) and date in same row
                {
                    \begin{tabularx}{\tabularxwidth}{@{}X r}
                        % First line - project name in bold, and skills/tech in brackets
                        \textbf{PaperRank Framework}
                            (Kubernetes, Python, Redis, SciPy, scikit-learn)
                        % Alignment separator (after if so it get triggered either way)
                        &
                        % Date on the right
                        \textbf{
        2018} \\
                    \end{tabularx}
                } \\

            % Collaborators
            

            % Awards
            

            % Printing website (if exists)
            
                \url{https://git.rukmal.me/PaperRank} \\
            

            % description (only using one)
            
    The PaperRank framework is designed to enable bibliometrics and citation analysis of academic literature graphs. It is highly extensible, and designed to be corpus-agnostic; currently, it is configured for use with the NCBI PubMed database. \\

        \end{tabularx}

        % Add vertical space if not last iteration
        
            \vspace{.5em}
        

    

        \begin{tabularx}{\tabularxwidth}{X}
            % Nested tabular to handle title (fancy, with skills) and date in same row
                {
                    \begin{tabularx}{\tabularxwidth}{@{}X r}
                        % First line - project name in bold, and skills/tech in brackets
                        \textbf{Derivative Visualizer}
                            (Knitr, R)
                        % Alignment separator (after if so it get triggered either way)
                        &
                        % Date on the right
                        \textbf{
        2017} \\
                    \end{tabularx}
                } \\

            % Collaborators
            

            % Awards
            

            % Printing website (if exists)
            
                \url{https://git.rukmal.me/DerivativeVisualizer} \\
            

            % description (only using one)
            
    An R library for plotting simple profit/loss graphs for equity and derivative positions. \\

        \end{tabularx}

        % Add vertical space if not last iteration
        
            \vspace{.5em}
        

    

        \begin{tabularx}{\tabularxwidth}{X}
            % Nested tabular to handle title (fancy, with skills) and date in same row
                {
                    \begin{tabularx}{\tabularxwidth}{@{}X r}
                        % First line - project name in bold, and skills/tech in brackets
                        \textbf{401(k) Portfolio Optimization Analysis}
                            (Knitr, LaTeX, R)
                        % Alignment separator (after if so it get triggered either way)
                        &
                        % Date on the right
                        \textbf{
        2016} \\
                    \end{tabularx}
                } \\

            % Collaborators
            

            % Awards
            

            % Printing website (if exists)
            
                \url{https://git.rukmal.me/UW-CFRM-462-Portfolio-Optimization} \\
            

            % description (only using one)
            
    An analysis of efficient portfolios of exchange traded funds, including time series modeling of returns with Monte Carlo simulations, and heteroskedasticity analysis. \\

        \end{tabularx}

        % Add vertical space if not last iteration
        
            \vspace{.5em}
        

    

        \begin{tabularx}{\tabularxwidth}{X}
            % Nested tabular to handle title (fancy, with skills) and date in same row
                {
                    \begin{tabularx}{\tabularxwidth}{@{}X r}
                        % First line - project name in bold, and skills/tech in brackets
                        \textbf{ZocialGPA Stack}
                            (Amazon Web Services, Docker, JavaScript, MongoDB, NLTK, Node.js, Python, Redis)
                        % Alignment separator (after if so it get triggered either way)
                        &
                        % Date on the right
                        \textbf{
        2015} \\
                    \end{tabularx}
                } \\

            % Collaborators
            

            % Awards
            

            % Printing website (if exists)
            
                \url{https://github.com/zocialgpa} \\
            

            % description (only using one)
            
    ZocialGPA was a multiplatform social analytics product, which used Facebook, Twitter, and LinkedIn feeds to compute a ZGPA score for users, based on a collection of statistical NLP heuristics. \\

        \end{tabularx}

        % Add vertical space if not last iteration
        

    


    % Awards and Certifications
    %---------------------------------------------

    \pagebreak[3]

    % Only print title if any of either exists
    % This is done differently because of the double-section thing
    
        \section{Awards and Certifications}

    % Global counter (over both award and certification; for spacing insertion)
    

    
        
            \begin{minipage}{\tabularxwidth}
            % iterating over awards and certifications
            \begin{tabularx}{\tabularxwidth}{X}
                % Nested tabular to handle title and date in same row
                {
                    \begin{tabularx}{\tabularxwidth}{@{}X r}
                        % Award/Certification name
                        \textbf{Provost's Master's Fellowship} &
                        % Date on the right
                        \textbf{
        2017} \\
                    \end{tabularx}
                } \\
                % Link (if any)
                    \url{https://www.stevens.edu/admissions/tuition-financial-aid/graduate-funding-aid/assistantships-fellowships} \\
                % descriptions
                
                    
    Awarded to first-year Graduate students entering the Stevens Institute of Technology for graduate study. \\
                
            \end{tabularx}

            % Add vertical space if not last iteration
            
                \vspace{.5em}
            

            % Increment counter
            

            \end{minipage}
        
    
        
            \begin{minipage}{\tabularxwidth}
            % iterating over awards and certifications
            \begin{tabularx}{\tabularxwidth}{X}
                % Nested tabular to handle title and date in same row
                {
                    \begin{tabularx}{\tabularxwidth}{@{}X r}
                        % Award/Certification name
                        \textbf{International Baccalaureate Diploma (The British School in Colombo)} &
                        % Date on the right
                        \textbf{
        2013} \\
                    \end{tabularx}
                } \\
                % Link (if any)
                    \url{https://www.ibo.org/programmes/diploma-programme/} \\
                % descriptions
                
                    
    Higher Level: Mathematics, Physics, Chemistry; Standard Level: Geography, English, Spanish; Extended Essay: Electromagnetism (Physics). \\
                
            \end{tabularx}

            % Add vertical space if not last iteration
            
                \vspace{.5em}
            

            % Increment counter
            

            \end{minipage}
        
            \begin{minipage}{\tabularxwidth}
            % iterating over awards and certifications
            \begin{tabularx}{\tabularxwidth}{X}
                % Nested tabular to handle title and date in same row
                {
                    \begin{tabularx}{\tabularxwidth}{@{}X r}
                        % Award/Certification name
                        \textbf{Bloomberg Market Concepts Certificate} &
                        % Date on the right
                        \textbf{
        2017} \\
                    \end{tabularx}
                } \\
                % Link (if any)
                    \url{https://www.bloomberg.com/professional/product/bloomberg-market-concepts/} \\
                % descriptions
                
                    
    A Bloomberg Terminal certification through the lens of Economic Indicators, Currencies, Fixed Income, and Equities. \\
                
            \end{tabularx}

            % Add vertical space if not last iteration
            
                \vspace{.5em}
            

            % Increment counter
            

            \end{minipage}
        
    



    % Activities
    %---------------------------------------------

    % Title only if exists
    
        \section{Activities}

    % Iterating over activity types
    
        \begin{tabularx}{\tabularxwidth}{X}
            % First line - activity type name in bold
            \textbf{Clubs and Organizations} \\
            % Listing all constituent activities
            
    
            Delta Upsilon Fraternity, 
            Foster Finance Association, 
            Husky Traders, 
            Stevens Society of Financial Engineers (SSFE), 
            UW Financial Engineering Club (UWFEC), 
            UW Hackers, 
            UW Sri Lanka Student Organization, 
            Washington Yacht Club \\
        \end{tabularx}

        % Add vertical space if not last iteration
        
            \vspace{.5em}
        

    
        \begin{tabularx}{\tabularxwidth}{X}
            % First line - activity type name in bold
            \textbf{Hackathons} \\
            % Listing all constituent activities
            
    
            Dubhacks, 
            Facebook Seattle Regional Hackathon, 
            IBM Bluehack Against Human Trafficking, 
            StartupUW Startup Weekend \\
        \end{tabularx}

        % Add vertical space if not last iteration
        
            \vspace{.5em}
        

    
        \begin{tabularx}{\tabularxwidth}{X}
            % First line - activity type name in bold
            \textbf{Recreational Sports} \\
            % Listing all constituent activities
            
    
            Basketball, 
            Running, 
            Strength Training, 
            Swimming \\
        \end{tabularx}

        % Add vertical space if not last iteration
        
            \vspace{.5em}
        

    
        \begin{tabularx}{\tabularxwidth}{X}
            % First line - activity type name in bold
            \textbf{Volunteering Activities} \\
            % Listing all constituent activities
            
    
            Conscious Crew Volunteer, 
            University District Food Bank Volunteer \\
        \end{tabularx}

        % Add vertical space if not last iteration
        

    

    % Contact Information
    %---------------------------------------------

    \section{Contact Information}
    
    % This is to change the width between the table columns
    \setlength{\tabcolsep}{18pt}

    \begin{center}
        \begin{tabular}{ll}
            6421 139th Pl NE Apt 50 & Phone: (206) 839-6891 \\
            Redmond, WA & Email: rukmal.weerawarana@gmail.com \\
            98052-4588 & Homepage: \url{https://rukmal.me} \\
            USA & LinkedIn: \url{https://linkedin.com/in/rukmalw}
        \end{tabular}
    \end{center}

\end{document}
