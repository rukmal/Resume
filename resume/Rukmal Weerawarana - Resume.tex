\documentclass[10pt]{article}

% Package Imports
%----------------

\usepackage{enumitem}  % Advanced list settings
\usepackage{fancyhdr}  % Fancy headers and footers
\usepackage[T1]{fontenc}  % Expanded font encoding
\usepackage{geometry}  % Page geometry (margins, etc.)
\usepackage{hyperref}  % Hyperlinks
\usepackage[none]{hyphenat}  % Hyphenation settings
\usepackage{multirow}  % Multirow table
\usepackage[defaultsans]{opensans}  % Opensans font
\usepackage[document]{ragged2e}  % Text alignment
\usepackage{tabularx}  % Advanced tables
\usepackage{titlesec}  % Title formatting
\usepackage[svgnames]{xcolor}  % Colors

% LaTeX Configuration
%---------------------

% Page margins
\geometry{top=.3in,
          bottom=.3in,
          left=.25in,
          right=.25in}

% Document font size
\newcommand{\cvfontsize}{9}

% Document font stuff
\renewcommand{\familydefault}{\sfdefault}
\renewcommand{\normalsize}{\fontsize{\cvfontsize}{\baselineskip}\selectfont}

% Highlight color
\newcommand{\highlightcolor}{RoyalBlue}

% Reformat section
\titleformat{\section}[block]
{\color{\highlightcolor} \Large \bf \scshape }
{}{0em}{}

% No indentation
\setlength{\parindent}{0pt}

% Table width
\newcommand{\tabularxwidth}{\textwidth}

% Link formatting
\hypersetup{
    colorlinks=true,
    linkcolor=blue,
    urlcolor=blue
}

% Link color command
\newcommand{\changeurlcolor}[1]{\hypersetup{urlcolor=#1}}

% Header and Footer Configuration
%--------------------------------

% Changing page style to fancy
\pagestyle{fancy}

% Applying default headers and footers
\fancyhf{}

% Renewing commands for the header and footer lines
\renewcommand{\headrulewidth}{0pt}
\renewcommand{\footrulewidth}{0pt}

% Changing offset from the margins
\setlength{\footskip}{0pt}

% Setting footer to be "Page #" on bottom right
\rfoot{Page \thepage}

% Setting last page footer to include last updated date, and Precis signature.
\AtEndDocument{\lfoot{Last updated December 15, 2021. Built with Precis (\url{https://precis.rukmal.me}).}}


% Jinja Macros
%--------------

% Get state or country (state takes preference)

% Get location in "City, State/Country" format

% Format date (to MMM 'YY, eg: "May '19"')

% Get date range

% Escape special characters in text
% Currently cleans: '$'

% Properly display the organization of a work Experience


% List the contents of a nested list of lists with a delimiter

% Only print title if it exists


% Expand a list of objects, chaining 'hasName' together with comma delimiter

% Format a portfolio date (i.e. just the year)



\begin{document}
    % Resume Header - Name, address, website and email
    %---------------------------------------------
    \begin{tabularx}{\textwidth}{@{}X r@{}}
        \multirow{3}{*}{\color{\highlightcolor} \huge \textbf{Rukmal Weerawarana}}
        & {\color{\highlightcolor} 6421 139th Pl NE Apt 50, Redmond, WA 98052-4588} \\
        & {\color{\highlightcolor} (206) 839-6891 $\; \bullet \;$ rukmal.weerawarana@gmail.com} \\
        & {\color{\highlightcolor} https://linkedin.com/in/rukmalw $\; \bullet \;$ https://rukmal.me} \\
    \end{tabularx}

    \vspace{.1em}

    \noindent{\rule{\linewidth}{.1em}}

    % Education
    %---------------------------------------------

    \vspace{-1em}

    
        \section{Education}
        \vspace{-.5em}

    % Iterating over degrees
    
        \begin{tabularx}{\tabularxwidth}{X r}
            % First line - degree title, and location
            \textbf{Master of Science (Financial Engineering)} & \textbf{
    Hoboken, 
        NJ} \\
            % Degree university, and location
            \quad \textcolor{\highlightcolor}{Stevens Institute of Technology} & 
    May ‘19 \\
            % Degree school and department
            \quad School of Business (Division of Financial Engineering) & \\
            % Degree GPA (if any)
            
                \quad \textit{GPA (4.0 Scale):} 3.8
                % Degree awards (if any)
                % NOTE: Only displayed if GPA is displayed too
                            | \textit{Awards:} 
    Provost's Master's Fellowship & \\
                        
                    
                    
                    
                
            
        \end{tabularx}

        % Add vertical space if not last iteration
        
            \vspace{.3em}
        

    
        \begin{tabularx}{\tabularxwidth}{X r}
            % First line - degree title, and location
            \textbf{Bachelor of Arts in Business Administration (Finance)} & \textbf{
    Seattle, 
        WA} \\
            % Degree university, and location
            \quad \textcolor{\highlightcolor}{University of Washington} & 
    Jun ‘17 \\
            % Degree school and department
            \quad Michael G. Foster School of Business (Department of Finance and Business Economics) & \\
            % Degree GPA (if any)
            
        \end{tabularx}

        % Add vertical space if not last iteration
        

    

    \vspace{-0.5em}

    % Work Experience
    %---------------------------------------------

    
        \section{Work Experience}
        \vspace{-.5em}

    
    % Put in minipage to prevent mid-entry pagebreak
    \begin{minipage}{\tabularxwidth}

        \begin{tabularx}{\tabularxwidth}{X r}
            % First line - job title and location
            \textbf{Research Collaborator} & \textbf{Troy, 
        NY} \\
            % Display employment organization row
            
    % Determining number of unique parent organizations
    
    
        % Last element of the list corresponds to the highest level (i.e. 'parent') orgNone
    

    
        % NOTE: This is for the case that there is no more than one base parent organization
        % Main employment organization and date range
        \textcolor{\highlightcolor}{Rensselaer Polytechnic Institute} & 
        
    Sep ‘21 - Present \\
        % Sub organizations (if any)
        
            % Top level sub organization once
            \textit{SCIENCE Blockchain Project;}
                \textit{Institute for Data Exploration and Applications (IDEA)} & \\
        
    
            % Other job titles
            
        \end{tabularx}

        % descriptions
        \begin{itemize}[noitemsep, topsep=3pt, parsep=0pt, partopsep=0pt]
            
                \item 
    Assist in the research and development process of a trustworthy, accountable data sharing ecosystem for biomedical research.
            
                \item 
    Lead efforts in developing cloud infrastructure to process and analyze data to build the underlying collaborator index. Currently exploring tokenomics policies for incentivizing data sharing in research networks.
            
                \item 
    Help mentor undergraduate students by providing assistance with research organization, infrastructure development, project implementation, and publication drafting.
            
        \end{itemize}

        % Add vertical space if not last iteration
        
            \vspace{.3em}
        

    \end{minipage}
    
    % Put in minipage to prevent mid-entry pagebreak
    \begin{minipage}{\tabularxwidth}

        \begin{tabularx}{\tabularxwidth}{X r}
            % First line - job title and location
            \textbf{Software Engineer (Data Science)} & \textbf{Seattle, 
        WA} \\
            % Display employment organization row
            
    % Determining number of unique parent organizations
    
    
        % Last element of the list corresponds to the highest level (i.e. 'parent') orgNone
    

    
        % NOTE: This is for the case that there is no more than one base parent organization
        % Main employment organization and date range
        \textcolor{\highlightcolor}{ExtraHop Networks} & 
        
    Sep ‘19 - 
    Feb ‘22 \\
        % Sub organizations (if any)
        
            % Top level sub organization once
            \textit{Unusual Behaviors Group;}
                \textit{Data Science R\&D} & \\
        
    
            % Other job titles
            
        \end{tabularx}

        % descriptions
        \begin{itemize}[noitemsep, topsep=3pt, parsep=0pt, partopsep=0pt]
            
                \item 
    Designed and developed high throughput data pipelines to transport and load semi structured data across multiple globally distributed data centers. Experienced in creating enriched data sets to deliver insights to internal and external stakeholders.
            
                \item 
    Produced numerous Data Science applications for both research and production use, with modern AWS serverless technologies and architectures. Applied modern Data Science algorithms and methodologies to Big Data to deliver insights to our customers.
            
                \item 
    Formalized, engineered, and managed the core internal ETL pipeline for over 2 years, delivering value to internal teams across the business. Scaled the core pipeline to handle an order of magnitude increase in data volume, while increasing functionality, due to increased adoption of ExtraHop's Cybersecurity product, Reveal(X) 360.
            
                \item 
    Enhanced ExtraHop's core threat hunting ability as a detector writer, applying unsupervised learning algorithms to Big Data scale traffic flows in the cloud. Implemented real-time detectors that act as early warning signals of potential bad actors on corporate networks across our customer base.
            
                \item 
    Participated in recruitment efforts, and interviewed teammates before, and during the COVID pandemic. Assisted with mentoring and on-boarding new team members.
            
        \end{itemize}

        % Add vertical space if not last iteration
        
            \vspace{.3em}
        

    \end{minipage}
    
    % Put in minipage to prevent mid-entry pagebreak
    \begin{minipage}{\tabularxwidth}

        \begin{tabularx}{\tabularxwidth}{X r}
            % First line - job title and location
            \textbf{Research Assistant} & \textbf{Hoboken, 
        NJ} \\
            % Display employment organization row
            
    % Determining number of unique parent organizations
    
    
        % Last element of the list corresponds to the highest level (i.e. 'parent') orgNone
    
        % Last element of the list corresponds to the highest level (i.e. 'parent') orgNone
    

    
        % NOTE: This is for the case that there is no more than one base parent organization
        % Main employment organization and date range
        \textcolor{\highlightcolor}{Stevens Institute of Technology} & 
        
    Aug ‘18 - 
    May ‘19 \\
        % Sub organizations (if any)
        
            % Top level sub organization once
            \textit{Sensorimotor Control Laboratory;}
                \textit{Stevens Institute for Artificial Intelligence} \textit{\&}
                \textit{Department of Biomedical Engineering} & \\
        
    
            % Other job titles
            
        \end{tabularx}

        % descriptions
        \begin{itemize}[noitemsep, topsep=3pt, parsep=0pt, partopsep=0pt]
            
                \item 
    Designed and implemented algorithms to assess and classify tremor severity in patients with late-stage Parkinson's Disease.
            
                \item 
    Created a highly scalable and extensible web application to be used by the researchers in the lab during this project. This web application incorporated HIPAA-compliant data storage and access, as well as efficient cluster management with Docker and Kubernetes.
            
        \end{itemize}

        % Add vertical space if not last iteration
        
            \vspace{.3em}
        

    \end{minipage}
    
    % Put in minipage to prevent mid-entry pagebreak
    \begin{minipage}{\tabularxwidth}

        \begin{tabularx}{\tabularxwidth}{X r}
            % First line - job title and location
            \textbf{Summer Research Fellow} & \textbf{Troy, 
        NY} \\
            % Display employment organization row
            
    % Determining number of unique parent organizations
    
    
        % Last element of the list corresponds to the highest level (i.e. 'parent') orgNone
    
        % Last element of the list corresponds to the highest level (i.e. 'parent') orgNone
    

    
        % NOTE: This is for the case that there is more than one base parent organization
        % Main employment organization and date range
        \textcolor{\highlightcolor}{RPI-IBM HEALS Research Center} & 
        
    May ‘18 - 
    Aug ‘18 \\
        % Parent organizations (if any)
        
            
                \textit{Tetherless World Constellation; Rensselaer Polytechnic Institute} & \\
            
                \textit{AI Horizons Network; IBM Research} & \\
            
        
    
            % Other job titles
            
        \end{tabularx}

        % descriptions
        \begin{itemize}[noitemsep, topsep=3pt, parsep=0pt, partopsep=0pt]
            
                \item 
    Led the design and development of the PaperRank Framework, a methodology for deriving probabilistic community trust in academic publications. PaperRank utilized the PageRank algorithm, coupled with a Gamma Mixture Model applied to citation networks of academic publications. A proof-of-concept was implemented, from extraction to final trust score computation, analyzing over 14 Million articles from the NCBI PubMed Database.
            
                \item 
    Formulated and implemented novel strategies for semantically-enhanced automated extraction of medical directives from Clinical Practice Guidelines (CPGs), for eventual inclusion in a knowledge graph of Diabetes diagnosis and treatment directives. Built the 'Guideline Explorer', a tool for efficiently visualizing and examining the American Diabetes Association's 2018 CPGs.
            
                \item 
    Explored the field of 'Semantalytics', which lies at the intersection of Semantics and Analytics. Drafted a Vision statement for the future exploration of this novel field of research, through the lens of bioinformatics.
            
        \end{itemize}

        % Add vertical space if not last iteration
        

    \end{minipage}
    

    \vspace{-0.5em}

    % Publications
    %---------------------------------------------

    
        \section{Publications}
        \vspace{-.5em}

    
        \begin{minipage}{\tabularxwidth}
        \begin{tabularx}{\tabularxwidth}{X}
            % Nested tabular to handle title and date in same column
            {
                \begin{tabularx}{\tabularxwidth}{@{}X r}
                    % Prefix with the current status (if any)
                    % Publication title
                    \textbf{\changeurlcolor{black}\href{https://arxiv.org/abs/1906.03935}{Learned Sectors: A fundamentals-driven sector reclassification project}} &
                    % Date on the right
                    \textbf{
        2019} \\
                \end{tabularx}
            } \\
            % Authors
            Rukmal Weerawarana, Yiyi Zhu, Yuzhen He \\

            % TODO: This part can be improved in the future
            % NOTE: Assuming that both publication & conference/journal exists if one does
            
                \textit{arXiv preprint; arXiv:1906.03935} \\
            
            % Printing DOI (if exists)
            

            % description (only using one)
            
    Market sectors play a key role in enabling the efficient flow of capital through the modern Global economy. An analysis of existing sectorization heuristics show that they are not entirely quantitatively driven, but rather are highly subjective and rooted in dogma. To this end, we introduce a new fundamentals-driven Learned Sectors heuristic.
        \end{tabularx}

        % Add vertical space if not last iteration
        
            \vspace{.3em}
        

        \end{minipage}
    
        \begin{minipage}{\tabularxwidth}
        \begin{tabularx}{\tabularxwidth}{X}
            % Nested tabular to handle title and date in same column
            {
                \begin{tabularx}{\tabularxwidth}{@{}X r}
                    % Prefix with the current status (if any)
                        \textit{(Draft) }
                    % Publication title
                    \textbf{\changeurlcolor{black}\href{https://drive.google.com/open?id=1SlSfZrwOQYP0mrKbrGLjjAFzWCk-qlwm}{Inferring Community Trust from Citation Graphs}} &
                    % Date on the right
                    \textbf{
        2019} \\
                \end{tabularx}
            } \\
            % Authors
            Jamie McCusker, Rukmal Weerawarana, Alexander New, Kristin P. Bennett, Deborah L. McGuinness \\

            % TODO: This part can be improved in the future
            % NOTE: Assuming that both publication & conference/journal exists if one does
            
            % Printing DOI (if exists)
            

            % description (only using one)
            
    We introduce the PaperRank scoring algorithm; a proxy of scientific community trust in a given publication. This score is derived from the classic PageRank algorithm (applied to academic citation networks), in conjunction with a one-dimensional Gamma Mixture Model to normalize the PageRank scores on a 3-group publication notoriety heuristic.
        \end{tabularx}

        % Add vertical space if not last iteration
        
            \vspace{.3em}
        

        \end{minipage}
    
        \begin{minipage}{\tabularxwidth}
        \begin{tabularx}{\tabularxwidth}{X}
            % Nested tabular to handle title and date in same column
            {
                \begin{tabularx}{\tabularxwidth}{@{}X r}
                    % Prefix with the current status (if any)
                    % Publication title
                    \textbf{\changeurlcolor{black}\href{https://tw.rpi.edu/web/doc/semantic_modeling_of_cohort}{Semantic Modeling of Cohort Descriptions in Research Studies}} &
                    % Date on the right
                    \textbf{
        2018} \\
                \end{tabularx}
            } \\
            % Authors
            Shruthi Chari, Rukmal Weerawarana, Oshani Seneviratne, Jamie McCusker, Deborah L. McGuinness, Amar Das \\

            % TODO: This part can be improved in the future
            % NOTE: Assuming that both publication & conference/journal exists if one does
            
                \textit{Knowledge Representation and Semantics Workshop; AMIA 2018 Annual Symposium} \\
            
            % Printing DOI (if exists)
            

            % description (only using one)
            
    This research addresses a key challenge faced by physicians using Clinical Practice Guideline recommendations; determining how well idiosyncratic cohort evidence generalizes to the greater clinical population.
        \end{tabularx}

        % Add vertical space if not last iteration
        
            \vspace{.3em}
        

        \end{minipage}
    
        \begin{minipage}{\tabularxwidth}
        \begin{tabularx}{\tabularxwidth}{X}
            % Nested tabular to handle title and date in same column
            {
                \begin{tabularx}{\tabularxwidth}{@{}X r}
                    % Prefix with the current status (if any)
                    % Publication title
                    \textbf{\changeurlcolor{black}\href{https://drive.google.com/open?id=19uND_fkRTd_m-i-SYBznq1wuCq1IvHhO}{What is a Knowledge Graph?}} &
                    % Date on the right
                    \textbf{
        2018} \\
                \end{tabularx}
            } \\
            % Authors
            Jamie McCusker, John S. Erickson, Katherine Chastain, Sabbir Rashid, Rukmal Weerawarana, Marcello Bax, Deborah L. McGuinness \\

            % TODO: This part can be improved in the future
            % NOTE: Assuming that both publication & conference/journal exists if one does
            
            % Printing DOI (if exists)
            

            % description (only using one)
            
    This work attempts to synthesize a clear and unambiguous definition of a 'Knowledge Graph' that conforms to current knowledge graph research, while constraining the research space that may be considered a knowledge graph.
        \end{tabularx}

        % Add vertical space if not last iteration
        

        \end{minipage}
    

    \vspace{-0.5em}

    % Projects
    %---------------------------------------------

    
        \section{Selected Projects}
        \vspace{-.5em}

    

        \begin{tabularx}{\tabularxwidth}{X}
            % Nested tabular to handle title (fancy, with skills) and date in same row
                {
                    \begin{tabularx}{\tabularxwidth}{@{}X r}
                        % First line - project name in bold
                        \textbf{Precis}
                        % Printing website (if exists)
                            | \url{https://precis.rukmal.me}
                        % Alignment separator (after if so it get triggered either way)
                        &
                        % Date on the right
                        \textbf{
        2021} \\
                    \end{tabularx}
                } \\

            % Collaborators
            

            % description (only using one)
            
    Precis is an Ontology for modeling personal professional metadata. The extended Precis toolkit also includes a Pythonic search API for the Ontology, a JSON data loader, and an extensible templating engine. \\

        \end{tabularx}

        % Add vertical space if not last iteration
        
            \vspace{.3em}
        

    

        \begin{tabularx}{\tabularxwidth}{X}
            % Nested tabular to handle title (fancy, with skills) and date in same row
                {
                    \begin{tabularx}{\tabularxwidth}{@{}X r}
                        % First line - project name in bold
                        \textbf{fe621}
                        % Printing website (if exists)
                            | \url{https://git.rukmal.me/FE-621-Homework}
                        % Alignment separator (after if so it get triggered either way)
                        &
                        % Date on the right
                        \textbf{
        2019} \\
                    \end{tabularx}
                } \\

            % Collaborators
            

            % description (only using one)
            
    fe621 is a Python library that provides functionality for lattice based derivative pricing models, exotic option picing, Monte Carlo simulations, numerical differentiation and integration, and optimization. \\

        \end{tabularx}

        % Add vertical space if not last iteration
        
            \vspace{.3em}
        

    

        \begin{tabularx}{\tabularxwidth}{X}
            % Nested tabular to handle title (fancy, with skills) and date in same row
                {
                    \begin{tabularx}{\tabularxwidth}{@{}X r}
                        % First line - project name in bold
                        \textbf{reIndexer}
                        % Printing website (if exists)
                            | \url{https://git.rukmal.me/reIndexer}
                        % Alignment separator (after if so it get triggered either way)
                        &
                        % Date on the right
                        \textbf{
        2019} \\
                    \end{tabularx}
                } \\

            % Collaborators
            

            % description (only using one)
            
    reIndexer is a research tool for the backtest-driven evaluation of different sectorization heuristics, using a system of synthetic ETFs, and efficient portfolios of those synthetic ETFs. \\

        \end{tabularx}

        % Add vertical space if not last iteration
        
            \vspace{.3em}
        

    

        \begin{tabularx}{\tabularxwidth}{X}
            % Nested tabular to handle title (fancy, with skills) and date in same row
                {
                    \begin{tabularx}{\tabularxwidth}{@{}X r}
                        % First line - project name in bold
                        \textbf{HTKG and NYPD-Compstat-LD}
                        % Printing website (if exists)
                            | \url{https://git.rukmal.me/NYPD-Compstat-LD}
                        % Alignment separator (after if so it get triggered either way)
                        &
                        % Date on the right
                        \textbf{
        2019} \\
                    \end{tabularx}
                } \\

            % Collaborators
            
                \textit{Collaborators:} Ryan Hartman, Ayush Kalla, Kovid Shukla, Sanket Saharkar
            

            % description (only using one)
            
    HTKG (Human Trafficking Knolwedge Graph), and NYPD-Compstat-LD (the main data ingestion engine) is a knowledge graph platform for linking suspected human trafficking advertisements with crime data from the NYPD to retroactively assess trends. \\

        \end{tabularx}

        % Add vertical space if not last iteration
        
            \vspace{.3em}
        

    

        \begin{tabularx}{\tabularxwidth}{X}
            % Nested tabular to handle title (fancy, with skills) and date in same row
                {
                    \begin{tabularx}{\tabularxwidth}{@{}X r}
                        % First line - project name in bold
                        \textbf{PaperRank Framework}
                        % Printing website (if exists)
                            | \url{https://git.rukmal.me/PaperRank}
                        % Alignment separator (after if so it get triggered either way)
                        &
                        % Date on the right
                        \textbf{
        2018} \\
                    \end{tabularx}
                } \\

            % Collaborators
            

            % description (only using one)
            
    The PaperRank framework is designed to enable bibliometrics and citation analysis of academic literature graphs. It is highly extensible, and designed to be corpus-agnostic; currently, it is configured for use with the NCBI PubMed database. \\

        \end{tabularx}

        % Add vertical space if not last iteration
        

    

    \vspace{-0.5em}

    % Talks
    %---------------------------------------------

    
        \section{Selected Talks}
        \vspace{-.5em}

    
        \begin{minipage}{\tabularxwidth}
        \begin{tabularx}{\tabularxwidth}{X}
            % Nested tabular to handle title and date in same column
            {
                \begin{tabularx}{\tabularxwidth}{@{}X r}
                    % Talk title
                    \textbf{Neural Ordinary Differential Equations} &
                    % Date on the right
                    \textbf{
        2020} \\
                \end{tabularx}
            } \\

            % Collaborators
            

            % Printing website (if exists)
            
                \url{https://drive.google.com/file/d/1fqVH6GJe1TcRyD6tL2cDUXkkkhyRq4vY} \\
            
            % Printing DOI (if exists)
            

            % description (only using one)
            
    A literature review of Neural Ordinary Differential Equations by Chen et al., a new family of deep neural network models that parameterizes the hidden state of a neural network.
        \end{tabularx}
        % Add vertical space if not last iteration
        
            \vspace{.3em}
        

        \end{minipage}
    
        \begin{minipage}{\tabularxwidth}
        \begin{tabularx}{\tabularxwidth}{X}
            % Nested tabular to handle title and date in same column
            {
                \begin{tabularx}{\tabularxwidth}{@{}X r}
                    % Talk title
                    \textbf{Knowledge Graph Fundamentals} &
                    % Date on the right
                    \textbf{
        2018} \\
                \end{tabularx}
            } \\

            % Collaborators
            

            % Printing website (if exists)
            
                \url{https://drive.google.com/file/d/1f21S_QZtm6aYiYLnXPA5opgsoBdIX7zX} \\
            
            % Printing DOI (if exists)
            

            % description (only using one)
            
    An overview of the fundamental technology powering modern knowledge graphs, focusing on the concepts of semantic data, ontologies, and inference.
        \end{tabularx}
        % Add vertical space if not last iteration
        
            \vspace{.3em}
        

        \end{minipage}
    
        \begin{minipage}{\tabularxwidth}
        \begin{tabularx}{\tabularxwidth}{X}
            % Nested tabular to handle title and date in same column
            {
                \begin{tabularx}{\tabularxwidth}{@{}X r}
                    % Talk title
                    \textbf{High Frequency Trading (HFT) - A Deep Dive} &
                    % Date on the right
                    \textbf{
        2017} \\
                \end{tabularx}
            } \\

            % Collaborators
            

            % Printing website (if exists)
            
                \url{https://drive.google.com/file/d/1I7JuZhVzsAT84xe88lLsbHbSIa-Ev7eF} \\
            
            % Printing DOI (if exists)
            

            % description (only using one)
            
    A deep dive into High Frequency Trading (HFT), covering market microstructure, exchange dynamics, regulatory implications, electronic order execution models, RegNMS, algorithmic trading, and popular HFT-driven strategies for exploiting arbitrage opportunities.
        \end{tabularx}
        % Add vertical space if not last iteration
        

        \end{minipage}
    

\end{document}